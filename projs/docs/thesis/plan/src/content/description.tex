\chapter{Disclaimer}

This document only describes the ideas that I plan to explore in my thesis. 
This is not the final project description for the thesis.

\chapter{Overview}

The idea behind the project is to establish the base for a game engine. 
What this means is that the end product is not something that can compete with the likes of Unity and Unreal engine.
The goal is to lay the foundation for something that could compete in the future.
This means that the project will need to make crutial decisions on what frameworks libraries to use, and how the core of the engine should be architected.

The planned engine is the spiritual successor to my old engine called ShadowEngine. 
This means that some parts of the old codebase will be used but every line will have to be rigorously checked.
This project is not a one person project, I'm planning on involving friends of mine. My responsibilities and the code that I write will constitute as the thesis.

\section{Technical information}
The project will be mainly written in C++, but some higher level functionality might be implemented in C# For windowing and interacting with the OS it will use SDL2.
The rendering API is going to be Vulkan.
Our main focus initially is Windows both for development and for running, as that is what all planned participants have access to.

\chapter{Main tasks}
\section{Build system}
The first question that needs decision for the project is what build system to use.
There are many different C++ build systems. The most common is Cmake and MSBuild. These are both quite capable.
\subsection{Cmake}
Cmake is an old powerful build system. it is capable of building almost anything. It is a generator for make files witch are even older. 
Make files and in turn Cmake is highly organized around the actual compilation commands that get run in the end.
This makes them harder to configure.
(Magical strings and bad docs)

\subsection{MSbuild}
MSbuild is Microsoft's build system, it is quite powerful and has really good integration with Visual Studio. 
It is also capable of integrating C# projects to the same workspace. 
Sadly MSbuild can only build C++ on Windows witch means we can't support other platforms in the future.
MSbuild is also not supported by other IDEs like Clion meaning it would be a total vendor lock in.

\subsection{Bazel}
Bazel is a little known build system as it is mostly used by large projects.
It was developed for Google's internal repository, as they keep all of their code in a single monolithic repository.
This means the Bazel is capable of building basically any language (C/C++, Java C# JS/TS, etc.).
Bazel is also capable of managing project dependencies without using Git Submodules.

\subsection{Decision}
After trying both Msbuild and Bazel the decision was to use Bazel.
Sadly the Clion Bazel plugin has problems on windows,
but theses might be fixable with a few PRs or just shipping our own version of the plugin, as it is available on Github